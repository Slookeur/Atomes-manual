\chapter{Keyboard shortcuts and command line options}

\section{Keyboard shortcuts}

\allkbd

\clearpage

\section{Command line options}

\newcommand{\atoc}{\textcolor{blue}{atomes}}
\newcommand{\xyz}{\textcolor{cyan}{{\bf{\texttt{-xyz}}}}}
\newcommand{\cif}{\textcolor{cyan}{{\bf{\texttt{-cif}}}}}
\newcommand{\pdb}{\textcolor{cyan}{{\bf{\texttt{-pdb}}}}}
\newcommand{\this}{\textcolor{brown}{{\bf{\texttt{this.f}}}}}
\newcommand{\that}{\textcolor{brown}{{\bf{\texttt{that.f}}}}}
\newcommand{\fawf}{\textcolor{red}{{\bf{\texttt{file.awf}}}}}

\atomes\ can be used from the command line, including using the following options: 
\begin{itemize}
\item General options:
\begin{itemize}
\item[] \aob{-h} or \aob{--help}: short help.
\item[] \aob{-v} or \aob{--version}: version information.
\end{itemize}
\item File options:
\begin{itemize}
\item[] \aob{-awf filename}: open \atomes\ workspace file. 
\item[] \aob{-apf filename}: open \atomes\ project file.  
\item[] \aob{-xyz filename}: open coordinates in XYZ format \cite{XYZ}. 
\item[] \aob{-pdb filename}: open coordinates in PDB format. 
\item[] \aob{-ent filename}: open coordinates in PDB format. 
\item[] \aob{-c3d filename}: open coordinates in Chem3D format \cite{Chem3D}. 
\item[] \aob{-cif filename}: open coordinates in CIF format. 
\item[] \aob{-trj filename}: open CPMD trajectory \cite{CPMD}. 
\item[] \aob{-xdatcar filename}: open VASP trajectory \cite{VASP}. 
\item[] \aob{-hist filename}: open DL\_POLY history trajectory \cite{DLPOLY}. 
\item[] \aob{-ipf filename}: open ISAACS project file \cite{ISAACS}. 
\end{itemize} 
\end{itemize}
\begin{enumerate}
\item If no file option(s) is (are) provided then \atomes\ will try to open the file(s) based on its (their) extension(s) 
(ex: a file which name end by \aob{.xyz} will be assumed to follow the XYZ structure). 
\item The structure of \aob{filename} is expected to follow the structure introduced with the option (ex: \aob{-xyz} for a file that contains 
coordinates in the XYZ format), but \aob{filename} is not required to have the corresponding extention (ex: \aob{.xyz}). 
\item Since \atomes\ has a single workspace only a single workspace file can be opened via the command line.
\end{enumerate}
\begin{itemize}
\item Examples:
\item[] Open all PDB files in the active directory: 
\begin{scri}{15}
user@localhost ~]\$ \atoc\ *.pdb
\end{scri}
\item[] Open a workspace file, and other coordinates files: 
\begin{scri}{15}
user@localhost ~]\$ \atoc\ \pdb\ this.f \fawf\ \cif\ that.f *.xyz
\end{scri}
\end{itemize}
\atomes\ will parse the command line, detect that a workspace file "\fawf" is to be opened and will open it first. \\
Then the other coordinate files will be opened and added to the workspace: 
\begin{itemize}
\item As enforced by the "\pdb" option, \aob{this.f} is supposed to follow the PDB format.
\item As enforced by the "\cif" option, \aob{that.f} is supposed to follow the CIF format.
\item All files with the \aob{.xyz} extension in the directory will be opened assuming that they follow the XYZ format. 
\end{itemize}
