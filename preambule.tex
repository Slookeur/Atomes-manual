\usepackage[utf8]{inputenc}
\usepackage[T1]{fontenc}
\usepackage{babel}

\usepackage{pifont}
% Les polices possibles (pour tout le document)
%\usepackage[bitstream-charter]{mathdesign}

%Font ps pour le pdf
\usepackage{pslatex}

%Hyperliens pour le pdf
\usepackage[pdfauthor   = {Sébastien\ Le\ Roux},
    pdftitle    = {Atomes\ user\ manual},
    pdfsubject = {Atomes\ user\ manual},
    pdfkeywords = {Atomes\ User\ Manual\ Guide\ Help},
    pdfcreator = {Latex+DviPDF},
    pdfproducer = {LaTeX+DviPDF},
    pdfstartview=FitV   % Ouverture avec ajustement de l'image
    dvips=true,         % Use hyperref with dvips
    colorlinks=true,    % Lien hypertext en couleur
    plainpages=false,   %
    pagebackref=true,   % Permet d'ajouter des liens retour dans la biblio ...
    backref=page,       % .. ces liens pointent vers les 'pages' des citations
    hyperindex=true,    % Ajoute des liens dans l'index
    linktocpage=true,   % Lien sur les numéros de page et non le text 
    breaklinks=true,    % Permet le retour à la ligne dans les liens trop longs
    urlcolor= blue,     % Couleur des liens externes
%    linkcolor= black,   % Couleur des liens internes
    bookmarks=true,     % Création des signets pour Acrobat
    bookmarksopen=false % Toute l'arborescence est dépliée à l'ouverture
]{hyperref}
%\usepackage[pageref]{backref}

%Insertion d'images
\usepackage{graphicx}
\DeclareGraphicsExtensions{.eps}

%Dessins latex
% Brouillon ? = afficher les images "false" ou seulement les cadres "true"
% effet sur tout le document
\newcommand{\ddst}{false}
%\usepackage{picins}

\usepackage{color}

% Mode verbatim avancé
\usepackage{alltt}

% Système de liste/énumération
\usepackage{pifont}
\usepackage{enumerate}

% Ecriture des mathématiques
\usepackage{amsmath}
\usepackage{amssymb}
\usepackage{amscd}
\usepackage{theorem}

\usepackage{pxfonts}

% tableaux
\usepackage{hhline}
%\usepackage{array}
\usepackage{multirow}
\usepackage{tabls}

%% Mise en page
\voffset         0.0cm
\hoffset         0.0cm
\textheight     24.0cm
\textwidth      16.0cm
\topmargin      -0.5cm
\oddsidemargin   0.0cm
\evensidemargin  0.0cm

% pages en landscape dans document portrait
\usepackage{lscape}

% Aspect de pages
\usepackage{setspace}
%\onehalfspacing
%\doublespacing
%\setstretch{3}

%\usepackage{fancyhdr,fancybox}
%\fancyhead{}
%\fancyhead[RO]{\scriptsize{\slshape\rightmark}}
%\fancyhead[LO]{\scriptsize{\thepage}}
%\fancyhead[RE]{\scriptsize{\thepage}}
%\fancyhead[LE]{\scriptsize{\slshape\leftmark}}
%\pagestyle{fancy}

% Numérotation pour les sous-sous-sections
\setcounter{secnumdepth}{3}
% Table des matières/figures/tables par chapitre
%\usepackage{minitoc}
% Pour placer des notes de bas de pages dans les titres
%\usepackage[stable]{footmisc}

%Définitions de théorèmes
\theoremstyle{plain}
\theoremheaderfont{\scshape}
\theorembodyfont{\normalfont\itshape}
\newtheorem{HK}{Théorème}

% Créer un environnement résumé
\def\abstract{
   \begin{center}
   \begin{minipage}{12cm}
   \begin{center}{\bf Résumé}\end{center}\par\small}
\def\endabstract{\par\end{minipage}\end{center}\vspace{1cm}}

% Bilblio:
% Bilblio:
\newcommand{\aap}{Astron. \& Astrophys.}
\newcommand{\aasup}{Astron. \& Astrophys. Suppl. Ser.}
\newcommand{\aj}{Astron. J.}
\newcommand{\aph}{Acta Phys.}
\newcommand{\act}{Acta Cryst.}
\newcommand{\actc}{Acta Cryst.}
\newcommand{\acta}{Acta Cryst. A}
\newcommand{\actb}{Acta Cryst. B}
\newcommand{\advp}{Adv. Phys.}
\newcommand{\ajp}{Amer. J. Phys.}
\newcommand{\ajm}{Amer. J. Math.}
\newcommand{\amsci}{Amer. Sci.}
\newcommand{\anofd}{Ann. Fluid Dyn.}
\newcommand{\am}{Ann. Math.}
\newcommand{\ap}{Ann. Phys. (NY)}
\newcommand{\adp}{Ann. Phys. (Leipzig)}
\newcommand{\ao}{Appl. Opt.}
\newcommand{\apl}{Appl. Phys. Lett.}
\newcommand{\app}{Astroparticle Phys.}
\newcommand{\apj}{Astrophys. J.}
\newcommand{\apjsup}{Astrophys. J. Suppl.}
\newcommand{\apss}{Astrophys. Space Sci.}
\newcommand{\araa}{Ann. Rev. Astron. Astrophys.}
\newcommand{\baas}{Bull. Amer. Astron. Soc.}
\newcommand{\baps}{Bull. Amer. Phys. Soc.}
\newcommand{\cmp}{Comm. Math. Phys.}
\newcommand{\cpam}{Commun. Pure Appl. Math.}
\newcommand{\cppcf}{Comm. Plasma Phys. \& Controlled Fusion}
\newcommand{\cpc}{Comp. Phys. Comm.}
\newcommand{\cms}{Comp. Mat. Sci.}
\newcommand{\cqg}{Class. Quant. Grav.}
\newcommand{\cra}{C. R. Acad. Sci. A}
\newcommand{\crv}{Chem. Rev.}
\newcommand{\cp}{Chem. Phys.}
\newcommand{\cma}{Chem. Mater.}
\newcommand{\epl}{Eur. Phys. Lett.}
\newcommand{\ejm}{Eur. J. Mineral.}
\newcommand{\fed}{Fusion Eng. \& Design}
\newcommand{\ft}{Fusion Tech.}
\newcommand{\grg}{Gen. Relativ. Gravit.}
\newcommand{\ieeens}{IEEE Trans. Nucl. Sci.}
\newcommand{\ieeeps}{IEEE Trans. Plasma Sci.}
\newcommand{\ijimw}{Interntl. J. Infrared \& Millimeter Waves}
\newcommand{\ip}{Infrared Phys.}
\newcommand{\irp}{Infrared Phys.}
\newcommand{\jap}{J. Appl. Phys.}
\newcommand{\jac}{J. Appl. Cryst.}
\newcommand{\jacs}{J. Am. Chem. Soc.}
\newcommand{\jamcs}{J. Am. Ceram. Soc.}
\newcommand{\jasa}{J. Acoust. Soc. America}
\newcommand{\jcp}{J. Chem. Phys.}
\newcommand{\jcop}{J. Comp. Phys.}
\newcommand{\jetp}{Sov. Phys.--JETP}
\newcommand{\jetpl}{JETP Lett.}
\newcommand{\jfe}{J. Fusion Energy}
\newcommand{\jfm}{J. Fluid Mech.}
\newcommand{\jmp}{J. Math. Phys.}
\newcommand{\jne}{J. Nucl. Energy}
\newcommand{\jnec}{J. Nucl. Energy, C: Plasma Phys., Accelerators, Thermonucl. Res.}
\newcommand{\jncs}{J. Non-Cryst. Solids.}
\newcommand{\jnm}{J. Nucl. Mat.}
\newcommand{\joam}{J. Optoelect. Adv. Mat.}
\newcommand{\jpcssp}{J. Phys. C: Solid State Phys.}
\newcommand{\jpc}{J. Phys. Chem.}
\newcommand{\jpcs}{J. Phys. Chem. Sol.}
\newcommand{\jpcm}{J. Phys.: Cond. Mat.}
\newcommand{\jpp}{J. Plasma Phys.}
\newcommand{\jpsj}{J. Phys. Soc. Japan}
\newcommand{\jqc}{J. Quant. Chem.}
\newcommand{\jssc}{J. Sol. Stat. Chem.}
\newcommand{\jsi}{J. Sci. Instrum.}
\newcommand{\jvst}{J. Vac. Sci. \& Tech.}
\newcommand{\mcp}{Mat. Chem. Phys.}
\newcommand{\nat}{Nature}
\newcommand{\nature}{Nature}
\newcommand{\nedf}{Nucl. Eng. \& Design/Fusion}
\newcommand{\nf}{Nucl. Fusion}
\newcommand{\nim}{Nucl. Inst. \& Meth.}
\newcommand{\nimpr}{Nucl. Inst. \& Meth. in Phys. Res.}
\newcommand{\np}{Nucl. Phys.}
\newcommand{\npb}{Nucl. Phys. B}
\newcommand{\ntf}{Nucl. Tech./Fusion}
\newcommand{\npbpc}{Nucl. Phys. B (Proc. Suppl.)}
\newcommand{\inc}{Nuovo Cimento}
\newcommand{\nc}{Nuovo Cimento}
\newcommand{\pcg}{Phys. Chem. Glasses}
\newcommand{\pf}{Phys. Fluids}
\newcommand{\pfa}{Phys. Fluids A: Fluid Dyn.}
\newcommand{\pfb}{Phys. Fluids B: Plasma Phys.}
\newcommand{\pl}{Phys. Lett.}
\newcommand{\pla}{Phys. Lett. A}
\newcommand{\plb}{Phys. Lett. B}
\newcommand{\prep}{Phys. Rep.}
\newcommand{\pnas}{Proc. Nat. Acad. Sci. USA}
\newcommand{\pp}{Phys. Plasmas}
\newcommand{\ppcf}{Plasma Phys. \& Controlled Fusion}
\newcommand{\phitrsl}{Philos. Trans. Roy. Soc. London}
\newcommand{\plmb}{Phil. Mag. B} 
\newcommand{\pml}{Phil. Mag. Lett.}
\newcommand{\pmm}{Phil. Mag.}
\newcommand{\prl}{Phys. Rev. Lett.}
\newcommand{\pr}{Phys. Rev.}
\newcommand{\physrev}{Phys. Rev.}
\newcommand{\pra}{Phys. Rev. A}
\newcommand{\prb}{Phys. Rev. B}
\newcommand{\prc}{Phys. Rev. C}
\newcommand{\prd}{Phys. Rev. D}
\newcommand{\pre}{Phys. Rev. E}
\newcommand{\ps}{Phys. Scripta}
\newcommand{\pstb}{Phys. Stat. Sol. b}
\newcommand{\procrsl}{Proc. Roy. Soc. London}
\newcommand{\rmp}{Rev. Mod. Phys.}
\newcommand{\rsi}{Rev. Sci. Inst.}
\newcommand{\rpp}{Rep. Prog. Phys.}
\newcommand{\science}{Science}
\newcommand{\sciam}{Sci. Am.}
\newcommand{\susc}{Surf. Sci}
\newcommand{\sam}{Stud. Appl. Math.}
\newcommand{\sjpp}{Sov. J. Plasma Phys.}
\newcommand{\spd}{Sov. Phys.--Doklady}
\newcommand{\sptp}{Sov. Phys.--Tech. Phys.}
\newcommand{\spu}{Sov. Phys.--Uspeki}
\newcommand{\skt}{Sky and Telesc.}
\newcommand{\ssi}{Solid State Ionics}
\newcommand{\ssc}{Solid State Com.}
\newcommand{\ssnmr}{Solid State Nuc. Mag. Res.}
\newcommand{\zfp}{Zs. f. Phys.}
\newcommand{\zk}{Z. Kristallogr.}


%\usepackage{chapterbib}

%\usepackage[square,comma,sort&compress]{natbib}
%\usepackage{hypernat}

% Algo XML
\usepackage{listings}

% Texte souligné
\usepackage[normalem]{ulem}

% Mise en forme des légendes
\usepackage[hang]{caption2}
%\usepackage[hang]{caption}
\renewcommand{\captionfont}{\it}
\renewcommand{\captionlabelfont}{\bf}
\renewcommand{\captionlabeldelim}{$\quad$}

%\usepackage[format=plain,labelfont=bf,up,textfont=it,up]{caption}
\newcommand{\red}[1]{\textcolor{red}{#1}}
\newcommand{\blue}[1]{\textcolor{blue}{#1}}
\newcommand{\green}[1]{\textcolor{green}{#1}}
\newcommand{\violet}[1]{\textcolor{violet}{#1}}
\newcommand{\pink}[1]{\textcolor{pink}{#1}}
\newcommand{\aob}[1]{"{\tt{#1}}"}

\definecolor{lg}{rgb}{0.95,0.95,0.95}

\newcounter{htmlkey}
\setcounter{htmlkey}{2}
\usepackage{keystroke}
\newcommand{\sclxml}{\vspace{-0.75cm}
\begin{table}[!b]
{\footnotesize{\tt{
\begin{scri}{13}
\mbf{?xml version="1.0" encoding="UTF-8"?}
\mbf{!-- Simple chemical library XML file --}
\mbf{scl-xml}
 \mbf{class}\key{Aromatics}\mbf{/class}
 \mbf{names}
  \mbf{library-name}\key{Caffeine}\mbf{/library-name}
  \mbf{iupac-name}\key{Caffeine}\mbf{/iupac-name}
  \mbf{other-names}
   \mbf{name}\key{Guaranine}\mbf{/name}
   \mbf{name}\key{Theine}\mbf{/name}
   \mbf{name}\key{Trimethylxanthine}\mbf{/name}
   \mbf{name}\key{1,3,7-Trimethyl-1H-purine-2,6(3H,7H)-dione}\mbf{/name}
  \mbf{/other-names}
 \mbf{/names}
 \mbf{chemistry}
  \mbf{atoms}\key{24}\mbf{/atoms}
  \mbf{species number\blue{=}\key{"4"}}
   \mbf{label id\blue{=}\key{"0"} num\blue{=}\key{"8"}}\key{C}\mbf{/label}
   \mbf{label id\blue{=}\key{"1"} num\blue{=}\key{"10"}}\key{H}\mbf{/label}
   \mbf{label id\blue{=}\key{"2"} num\blue{=}\key{"2"}}\key{O}\mbf{/label}
   \mbf{label id\blue{=}\key{"3"} num\blue{=}\key{"4"}}\key{N}\mbf{/label}
  \mbf{/species}
 \mbf{/chemistry}
 \mbf{coordinates}
   \mbf{atom id\blue{=}\key{"1"} sp\blue{=}\key{"0"} x\blue{=}\key{"1.785021"} y\blue{=}\key{"-0.779129"} z\blue{=}\key{"-0.255949"}/}
   \mbf{atom id\blue{=}\key{"2"} sp\blue{=}\key{"0"} x\blue{=}\key{"0.401929"} y\blue{=}\key{"1.318216"} z\blue{=}\key{"-0.016004"}/}
   \mbf{atom id\blue{=}\key{"3"} sp\blue{=}\key{"0"} x\blue{=}\key{"-0.733464"} y\blue{=}\key{"0.413478"} z\blue{=}\key{"0.038347"}/}
   \mbf{atom id\blue{=}\key{"4"} sp\blue{=}\key{"0"} x\blue{=}\key{"-0.617909"} y\blue{=}\key{"-0.973998"} z\blue{=}\key{"-0.109571"}/}
   \mbf{atom id\blue{=}\key{"5"} sp\blue{=}\key{"0"} x\blue{=}\key{"-2.772439"} y\blue{=}\key{"-0.562269"} z\blue{=}\key{"0.209813"}/}
   \mbf{atom id\blue{=}\key{"6"} sp\blue{=}\key{"0"} x\blue{=}\key{"0.735608"} y\blue{=}\key{"-3.069196"} z\blue{=}\key{"-0.210904"}/}
   \mbf{atom id\blue{=}\key{"7"} sp\blue{=}\key{"0"} x\blue{=}\key{"-2.730788"} y\blue{=}\key{"1.961282"} z\blue{=}\key{"0.459758"}/}
   \mbf{atom id\blue{=}\key{"8"} sp\blue{=}\key{"0"} x\blue{=}\key{"2.908891"} y\blue{=}\key{"1.458603"} z\blue{=}\key{"-0.240195"}/}
   \mbf{atom id\blue{=}\key{"9"} sp\blue{=}\key{"1"} x\blue{=}\key{"-3.848274"} y\blue{=}\key{"-0.720508"} z\blue{=}\key{"0.332764"}/}
   \mbf{atom id\blue{=}\key{"10"} sp\blue{=}\key{"1"} x\blue{=}\key{"1.561100"} y\blue{=}\key{"-3.446348"} z\blue{=}\key{"-0.829847"}/}
   \mbf{atom id\blue{=}\key{"11"} sp\blue{=}\key{"1"} x\blue{=}\key{"-0.192018"} y\blue{=}\key{"-3.562176"} z\blue{=}\key{"-0.530166"}/}
   \mbf{atom id\blue{=}\key{"12"} sp\blue{=}\key{"1"} x\blue{=}\key{"0.935512"} y\blue{=}\key{"-3.340819"} z\blue{=}\key{"0.835372"}/}
   \mbf{atom id\blue{=}\key{"13"} sp\blue{=}\key{"1"} x\blue{=}\key{"-2.333498"} y\blue{=}\key{"2.706064"} z\blue{=}\key{"-0.245137"}/}
   \mbf{atom id\blue{=}\key{"14"} sp\blue{=}\key{"1"} x\blue{=}\key{"-2.525360"} y\blue{=}\key{"2.320312"} z\blue{=}\key{"1.478733"}/}
   \mbf{atom id\blue{=}\key{"15"} sp\blue{=}\key{"1"} x\blue{=}\key{"-3.818507"} y\blue{=}\key{"1.892617"} z\blue{=}\key{"0.326144"}/}
   \mbf{atom id\blue{=}\key{"16"} sp\blue{=}\key{"1"} x\blue{=}\key{"3.674681"} y\blue{=}\key{"0.991531"} z\blue{=}\key{"-0.874770"}/}
   \mbf{atom id\blue{=}\key{"17"} sp\blue{=}\key{"1"} x\blue{=}\key{"3.303682"} y\blue{=}\key{"1.537341"} z\blue{=}\key{"0.782105"}/}
   \mbf{atom id\blue{=}\key{"18"} sp\blue{=}\key{"1"} x\blue{=}\key{"2.721889"} y\blue{=}\key{"2.471199"} z\blue{=}\key{"-0.622647"}/}
   \mbf{atom id\blue{=}\key{"19"} sp\blue{=}\key{"2"} x\blue{=}\key{"2.892437"} y\blue{=}\key{"-1.297551"} z\blue{=}\key{"-0.212698"}/}
   \mbf{atom id\blue{=}\key{"20"} sp\blue{=}\key{"2"} x\blue{=}\key{"0.363733"} y\blue{=}\key{"2.532611"} z\blue{=}\key{"0.115710"}/}
   \mbf{atom id\blue{=}\key{"21"} sp\blue{=}\key{"3"} x\blue{=}\key{"1.659632"} y\blue{=}\key{"0.660920"} z\blue{=}\key{"-0.271271"}/}
   \mbf{atom id\blue{=}\key{"22"} sp\blue{=}\key{"3"} x\blue{=}\key{"0.612806"} y\blue{=}\key{"-1.608906"} z\blue{=}\key{"-0.393730"}/}
   \mbf{atom id\blue{=}\key{"23"} sp\blue{=}\key{"3"} x\blue{=}\key{"-2.109760"} y\blue{=}\key{"0.657267"} z\blue{=}\key{"0.234906"}/}
   \mbf{atom id\blue{=}\key{"24"} sp\blue{=}\key{"3"} x\blue{=}\key{"-1.874903"} y\blue{=}\key{"-1.560540"} z\blue{=}\key{"-0.000764"}/}
 \mbf{/coordinates}
\mbf{/scl-xml}
\end{scri}
}}}
\caption{\label{sml}Example of \aob{sml} file in XML format for caffeine.}
\end{table}
}
\newcommand{\sglxml}{\begin{table}[!h]
{\tiny{\tt{
\begin{scri}{12}
\mbf{?xml version="1.0" encoding="UTF-8"?}
\mbf{!-- Space group info XML file --}
\mbf{sg-xml}
  \mbf{space-group}\key{R-3c}\mbf{/space-group}
  \mbf{sg-num}\key{167}\mbf{/sg-num}
  \mbf{hm-symbol}\key{R -3 2/c}\mbf{/hm-symbol}
  \mbf{bravais}\key{Trigonal}\mbf{/bravais}
  \mbf{settings num\blue{=}\key{"2"}}
    \mbf{set name\blue{=}\key{"R_-3_c_:h"} x\blue{=}\key{"a"} y\blue{=}\key{"b"} z\blue{=}\key{"c"}}
      \mbf{points num\blue{=}\key{"3"}}
        \mbf{pt x\blue{=}\key{"0"} y\blue{=}\key{"0"} z\blue{=}\key{"0"}/}
        \mbf{pt x\blue{=}\key{"2/3"} y\blue{=}\key{"1/3"} z\blue{=}\key{"1/3"}/}
        \mbf{pt x\blue{=}\key{"1/3"} y\blue{=}\key{"2/3"} z\blue{=}\key{"2/3"}/}
      \mbf{/points}
    \mbf{/set}
    \mbf{set name\blue{=}\key{"R_-3_c_:r"} x\blue{=}\key{"2/3a+1/3b+1/3c"} y\blue{=}\key{"-1/3a+1/3b+1/3c"} z\blue{=}\key{"-1/3a-2/3b+1/3c"}}
      \mbf{points num\blue{=}\key{"3"}}
        \mbf{pt x\blue{=}\key{"0"} y\blue{=}\key{"0"} z\blue{=}\key{"0"}/}
        \mbf{pt x\blue{=}\key{"0"} y\blue{=}\key{"0"} z\blue{=}\key{"0"}/}
        \mbf{pt x\blue{=}\key{"0"} y\blue{=}\key{"0"} z\blue{=}\key{"0"}/}
      \mbf{/points}
    \mbf{/set}
  \mbf{/settings}
  \mbf{wyckoff num\blue{=}\key{"6"}}
    \mbf{wyck id\blue{=}\key{"1"} mul\blue{=}\key{"12"} let\blue{=}\key{"f"} site\blue{=}\key{"1"}}
      \mbf{pos x\blue{=}\key{"x"} y\blue{=}\key{"y"} z\blue{=}\key{"z"}/}
      \mbf{pos x\blue{=}\key{"-y"} y\blue{=}\key{"x-y"} z\blue{=}\key{"z"}/}
      \mbf{pos x\blue{=}\key{"-x+y"} y\blue{=}\key{"-x"} z\blue{=}\key{"z"}/}
      \mbf{pos x\blue{=}\key{"y"} y\blue{=}\key{"x"} z\blue{=}\key{"-z+1/2"}/}
      \mbf{pos x\blue{=}\key{"x-y"} y\blue{=}\key{"-y"} z\blue{=}\key{"-z+1/2"}/}
      \mbf{pos x\blue{=}\key{"-x"} y\blue{=}\key{"-x+y"} z\blue{=}\key{"-z+1/2"}/}
      \mbf{pos x\blue{=}\key{"-x"} y\blue{=}\key{"-y"} z\blue{=}\key{"-z"}/}
      \mbf{pos x\blue{=}\key{"y"} y\blue{=}\key{"-x+y"} z\blue{=}\key{"-z"}/}
      \mbf{pos x\blue{=}\key{"x-y"} y\blue{=}\key{"x"} z\blue{=}\key{"-z"}/}
      \mbf{pos x\blue{=}\key{"-y"} y\blue{=}\key{"-x"} z\blue{=}\key{"z+1/2"}/}
      \mbf{pos x\blue{=}\key{"-x+y"} y\blue{=}\key{"y"} z\blue{=}\key{"z+1/2"}/}
      \mbf{pos x\blue{=}\key{"x"} y\blue{=}\key{"x-y"} z\blue{=}\key{"z+1/2"}/}
    \mbf{/wyck}
    \mbf{wyck id\blue{=}\key{"2"} mul\blue{=}\key{"6"} let\blue{=}\key{"e"} site\blue{=}\key{".2"}}
      \mbf{pos x\blue{=}\key{"x"} y\blue{=}\key{"0"} z\blue{=}\key{"1/4"}/}
      \mbf{pos x\blue{=}\key{"0"} y\blue{=}\key{"x"} z\blue{=}\key{"1/4"}/}
      \mbf{pos x\blue{=}\key{"-x"} y\blue{=}\key{"-x"} z\blue{=}\key{"1/4"}/}
      \mbf{pos x\blue{=}\key{"-x"} y\blue{=}\key{"0"} z\blue{=}\key{"3/4"}/}
      \mbf{pos x\blue{=}\key{"0"} y\blue{=}\key{"-x"} z\blue{=}\key{"3/4"}/}
      \mbf{pos x\blue{=}\key{"x"} y\blue{=}\key{"x"} z\blue{=}\key{"3/4"}/}
    \mbf{/wyck}
    \mbf{wyck id\blue{=}\key{"3"} mul\blue{=}\key{"6"} let\blue{=}\key{"d"} site\blue{=}\key{"-1"}}
      \mbf{pos x\blue{=}\key{"1/2"} y\blue{=}\key{"0"} z\blue{=}\key{"0"}/}
      \mbf{pos x\blue{=}\key{"0"} y\blue{=}\key{"1/2"} z\blue{=}\key{"0"}/}
      \mbf{pos x\blue{=}\key{"1/2"} y\blue{=}\key{"1/2"} z\blue{=}\key{"0"}/}
      \mbf{pos x\blue{=}\key{"0"} y\blue{=}\key{"1/2"} z\blue{=}\key{"1/2"}/}
      \mbf{pos x\blue{=}\key{"1/2"} y\blue{=}\key{"0"} z\blue{=}\key{"1/2"}/}
      \mbf{pos x\blue{=}\key{"1/2"} y\blue{=}\key{"1/2"} z\blue{=}\key{"1/2"}/}
    \mbf{/wyck}
    \mbf{wyck id\blue{=}\key{"4"} mul\blue{=}\key{"4"} let\blue{=}\key{"c"} site\blue{=}\key{"3."}}
      \mbf{pos x\blue{=}\key{"0"} y\blue{=}\key{"0"} z\blue{=}\key{"z"}/}
      \mbf{pos x\blue{=}\key{"0"} y\blue{=}\key{"0"} z\blue{=}\key{"-z+1/2"}/}
      \mbf{pos x\blue{=}\key{"0"} y\blue{=}\key{"0"} z\blue{=}\key{"-z"}/}
      \mbf{pos x\blue{=}\key{"0"} y\blue{=}\key{"0"} z\blue{=}\key{"z+1/2"}/}
    \mbf{/wyck}
    \mbf{wyck id\blue{=}\key{"5"} mul\blue{=}\key{"2"} let\blue{=}\key{"b"} site\blue{=}\key{"-3."}}
      \mbf{pos x\blue{=}\key{"0"} y\blue{=}\key{"0"} z\blue{=}\key{"0"}/}
      \mbf{pos x\blue{=}\key{"0"} y\blue{=}\key{"0"} z\blue{=}\key{"1/2"}/}
    \mbf{/wyck}
    \mbf{wyck id\blue{=}\key{"6"} mul\blue{=}\key{"2"} let\blue{=}\key{"a"} site\blue{=}\key{"32"}}
      \mbf{pos x\blue{=}\key{"0"} y\blue{=}\key{"0"} z\blue{=}\key{"1/4"}/}
      \mbf{pos x\blue{=}\key{"0"} y\blue{=}\key{"0"} z\blue{=}\key{"3/4"}/}
    \mbf{/wyck}
  \mbf{/wyckoff}
\mbf{/sg-xml}
\end{scri}
}}}
\caption{\label{sgl}Example of the \aob{167-R-3c.sgl} file in XML format for the R$\bar{3}$c space group.}
\end{table}
}

\newcommand{\mbf}[1]{<#1>}
\newcommand{\key}[1]{\red{#1}}

\newcommand{\isaacs}{I.S.A.A.C.S.}
\newcommand{\ISAACS}{Interactive Structure Analysis of Amorphous and Crystalline Systems}
\newcommand{\atomes}{{\em{\bf{Atomes}}}}
\newcommand{\activp}{{\em{\bf{active}}}}
\newcommand{\rpc}{$R_C(n)$}
\newcommand{\rpn}{$R_N(n)$}
\newcommand{\pnr}{$P_N(n)$}
\newcommand{\pnrmin}{$P_{N_{\text{min}}}(n)$}
\newcommand{\pnrmax}{$P_{N_{\text{max}}}(n)$}
\newcommand{\pmin}{$P_{\text{min}}(n)$}
\newcommand{\pmax}{$P_{\text{max}}(n)$}
\newcommand{\smin}{$s_{min}$}
\newcommand{\smax}{$s_{max}$}

\newcommand{\ges}{GeS$_2$}
\newcommand{\sio}{SiO$_2$}
\newcommand{\nn}{rings with $n$ nodes}
\newcommand{\con}{connectivity}
\newcommand{\conp}{connectivity profile}
\newcommand{\rstat}{ring statistics}

\newcommand{\dlpoly}{\href{https://www.scd.stfc.ac.uk/Pages/DL\_POLY.aspx}{DL-POLY}}
\newcommand{\lammps}{\href{https://lammps.sandia.gov/}{LAMMPS}}
\newcommand{\cpmd}{\href{http://www.cpmd.org}{CPMD}}
\newcommand{\cptk}{\href{http://cp2k.berlios.de}{CP2K}}

\newcommand{\atomesweb}{https://atomes.ipcms.fr}

\renewcommand{\figurename}{Figure}
\renewcommand{\tablename}{Table}
\newcommand{\laf}{\\}
\newcommand{\image}[2]{\includegraphics*[width=#1cm, keepaspectratio=true, draft=\ddst]{#2}}

\newcommand{\myfigure}[5]{\begin{figure}[!#1]{\hypertarget{#2}{\begin{center}{#3\caption[#4]{#5}\label{#2}}\end{center}}}\end{figure}}
\newcommand{\mytable}[5]{\begin{table}[!#1]{\begin{center}{#3\caption[#4]{#5}\label{#2}}\end{center}}\end{table}}

\newsavebox{\cobox}
\def\script{
  \noindent \laf \laf
  \begin{lrbox}
  \cobox
  \begin{minipage}[l]{16cm}
  \begin{alltt}}
\def\endscript{
  \end{alltt}
  \end{minipage}
  \end{lrbox}
  \colorbox{lg}{\usebox{\cobox}}
  \vspace{0.125cm}\par\noindent}

\def\scri#1{
  \noindent \laf \laf
  \begin{lrbox}
  \cobox
  \begin{minipage}[l]{#1cm}
  \begin{alltt}}
\def\endscri{
  \end{alltt}
  \end{minipage}
  \end{lrbox}
  \colorbox{lg}{\usebox{\cobox}}
  \vspace{0.25cm}\par\noindent}

% This file contains figures setup that have issues to be rendered in HTML
% Either multiple images and/or tabular layout.

% Chap 3

\newcommand{\mainfig}{
\myfigure{h}{main}{\image{15}{img/main/atomes-main}
\begin{tabular}{lclclclc}
{\bf{a)}} & & \hspace{-1cm} {\bf{b)}} & & \hspace{-1cm} {\bf{c)}} & & \hspace{-1cm} {\bf{d)}}\\
 & \hspace{-0.5cm} \image{4}{img/main/menu/m_workspace} &
 & \hspace{-1cm} \image{4}{img/main/menu/m_edit} &
 & \hspace{-1cm} \image{4}{img/main/menu/m_analyze} &
 & \hspace{-1cm} \image{4}{img/main/menu/m_help}
\end{tabular}}
{Main window of the \atomes\ program.}{Main window of the \atomes\ program.}}

\newcommand{\treefig}{
\myfigure{h}{ptree}{\begin{tabular}{ccccc}
 \hspace{-1.5cm}
 \image{5}{img/p-tree/p-open} & \raisebox{6cm}{$\Longrightarrow$} &
 \image{5}{img/p-tree/p-tree} & \raisebox{6cm}{$\Longrightarrow$} &
 \image{5}{img/p-tree/p-tree-full}
\end{tabular}}{Project tree in the \atomes\ program.}{Project tree in the \atomes\ program.}}

\newcommand{\bpdfig}{
\myfigure{h}{bpd}{\begin{tabular}{cc}
\image{8}{img/main/atomes-bpd} &
\image{8}{img/main/atomes-bpd2}
\end{tabular}}{The \aob{Box and periodicity} dialog in the \atomes\ program.}{The \aob{Box and periodicity} dialog in the \atomes\ program.}}

\newcommand{\oapfig}{
\myfigure{h}{oapf}{
\begin{tabular}{ccc}
\hspace{-1.5cm}
\image{9}{img/main/atomes-apf-1} & \raisebox{2.5cm}{$\Longrightarrow$} & 
\image{8}{img/main/atomes-apf-2}
\end{tabular}}
{The \aob{Open Project File(s)} dialog in the \atomes\ program.}
{The \aob{Open Project File(s)} dialog in the \atomes\ program.}}

\newcommand{\owpfig}{
\myfigure{h}{owpf}{
\begin{tabular}{ccc}
\hspace{-1.5cm}
\image{9}{img/main/atomes-awf-1} & \raisebox{2.5cm}{$\Longrightarrow$} & 
\image{8}{img/main/atomes-awf-2}
\end{tabular}}
{The \aob{Open Workspace} dialog in the \atomes\ program.}
{The \aob{Open Workspace} dialog in the \atomes\ program.}}

% Chap 4

\newcommand{\toolfig}{
\myfigure{h}{tools2}{
\begin{tabular}{cp{0.5cm}cp{0.5cm}c}
\hspace{-1cm}\image{5}{img/main/atomes-tools-2} & \raisebox{2.0cm}{$\Longrightarrow$} &
\image{5}{img/main/atomes-tools-3} & \raisebox{2.0cm}{$\Longrightarrow$} &
\image{5}{img/main/atomes-tools-4} \\
\end{tabular}
\begin{tabular}{cc}
\image{12}{img/curves/gr} & \raisebox{10cm}{$\Swarrow$}
\end{tabular}}
{Interaction with the \aob{Toolboxes} dialog in the \atomes\ program.}
{Interaction with the \aob{Toolboxes} dialog in the \atomes\ program.}}

\newcommand{\dmenufig}{
\myfigure{h}{dmenu}{
\image{13}{img/curves/data-menu-open}
\begin{tabular}{cc}
 & $\Swarrow$ \\
\includegraphics*[height=11.5cm, keepaspectratio=true, draft=\ddst]{img/curves/edit-win} &
\includegraphics*[height=11.5cm, keepaspectratio=true, draft=\ddst]{img/curves/edit-menu}
\end{tabular}}
{\aob{Data} and \aob{Edit Data} menus, \aob{Data Edition} window.}
{Top: The \aob{Data} and the \aob{Edit Data} menus in the top menu of the graph window. Bottom: the \aob{Data Edition} window and the associated contextual menu.}}

\newcommand{\egrfig}{
\myfigure{h}{egr}{
\hspace{-1cm}
\image{14}{img/curves/GeSe} \\ 
\image{14}{img/curves/rc-edit}}
{Examples of data plot edition in the \atomes\ program.}
{Examples of data plot edition in the \atomes\ program.}}

\newcommand{\setsfig}{
\myfigure{h}{m-sets}{
\begin{tabular}{lc}
{\bf{a)}} \\
  & \image{11}{img/curves/edit-set} \\
{\bf{b)}} \\
  & \image{11}{img/curves/add-set} \\
{\bf{c)}} \\
  & \image{11}{img/curves/remove-set}
\end{tabular}}
{The \aob{Edit Data} a), \aob{Add Data Set} b) and \aob{Remove Data Set} c) menus.}
{The \aob{Edit Data} a), \aob{Add Data Set} b) and \aob{Remove Data Set} c) submenus in the graph window contextual menu for the example workspace [Fig.~\ref{lGeSe}].}}

\newcommand{\zoomfig}{
\myfigure{h}{lzoom}{\begin{tabular}{lcr}
\image{6.5}{img/curves/1a} & \raisebox{2.5cm}{$\Longrightarrow$} &
\image{6.5}{img/curves/1b} \\
\image{6.5}{img/curves/2a} &  \raisebox{2.5cm}{$\Longrightarrow$} &
\image{6.5}{img/curves/2b} \\
\image{6.5}{img/curves/3a} &  \raisebox{2.5cm}{$\Longrightarrow$} &
\image{6.5}{img/curves/3b} \\
\image{6.5}{img/curves/4a} &  \raisebox{2.5cm}{$\Longrightarrow$} &
\image{6.5}{img/curves/4b} \\
\end{tabular}}
{The mouse left button zoom in/out of the graph window in the \atomes\ program.}
{The mouse left button zoom in/out of the graph window in the \atomes\ program.}}

% Chap 5

\newcommand{\moglfig}{
\myfigure{h}{mogl}{
\begin{tabular}{lccr}
 & {\bf{a)}} \\
\hspace{-1.5cm}\multirow{8}{5cm}{\image{5}{img/visu/menu/m_opengl}} & & \multicolumn{2}{r}{\image{12}{img/visu/menu/opengl/style}} \\
 &  {\bf{b)}} \\
& & \multicolumn{2}{r}{\image{12}{img/visu/menu/opengl/acmap}} \\
& & \image{12}{img/visu/menu/opengl/pcmap} \\
 & {\bf{c)}} \\
& & \multicolumn{2}{r}{\image{12}{img/visu/menu/opengl/render}} \\
 & {\bf{d)}} \\
& &  \multicolumn{2}{r}{\image{12}{img/visu/menu/opengl/qual}} 
\end{tabular}}{The \aob{OpenGL} menu and the attached submenus.}{The \aob{OpenGL} menu and the attached submenus.}}

\newcommand{\cmapfig}{
\myfigure{h}{cmaps}{
\begin{tabular}{lp{0.5cm}l}
\multicolumn{3}{c}{\image{8}{img/visu/cmap/cm-sp-b}} \\
 \\
\hspace{-1cm} {\bf{a)}} & & {\bf{b)}} \\
\hspace{-1cm} \image{8}{img/visu/cmap/cm-tc-b} & & \image{8}{img/visu/cmap/cm-pc-b} \\
\\
\hspace{-1cm} {\bf{c)}} & & {\bf{d)}} \\
\hspace{-1cm} \image{8}{img/visu/cmap/cm-frag-b} & & \image{8}{img/visu/cmap/cm-mol-b} \\
\end{tabular}}
{Standard color maps in the \atomes\ program.}
{The standard Colors Maps (CM) for the atoms accessible using the \aob{Color scheme(s)} menu. 
Top standard colors using the chemical species, 
{\bf{a)}} Total coordinations as CM, {\bf{b)}} Partial coordinations as CM, {\bf{c)}} Fragments as CM and {\bf{d)}} Molecules as CM.}}

\newcommand{\modlfig}{
\myfigure{h}{modl}{
\begin{tabular}{lccr}
 & {\bf{a)}} \\
& & \multicolumn{2}{r}{\image{9.2}{img/visu/menu/model/show}} \\
\multirow{8}{5cm}{\image{5}{img/visu/menu/m_model}} & & \multicolumn{2}{r}{\image{9.2}{img/visu/menu/model/sp_colors}} \\
 & & \multicolumn{2}{r}{\image{9.2}{img/visu/menu/model/labels}} \\
 & {\bf{b)}} \\
& & & \image{9.2}{img/visu/menu/model/ncutoffs} \\
 & {\bf{c)}} \\
& & \multicolumn{2}{r}{\image{9.2}{img/visu/menu/model/clones}} \\
 & {\bf{d)}} \\
& &  \multicolumn{2}{r}{\image{9.2}{img/visu/menu/model/box}} 
\end{tabular}}
{The \aob{Model} menu and the attached submenus.}{The \aob{Model} menu and the attached submenus.}}

\newcommand{\clonefig}{
\begin{figure}[!h]
\hypertarget{atcl}{
\hspace{-1.5cm}
\begin{tabular}{lcp{0.5cm}lc}
{\bf{a)}} & & & {\bf{b)}} \\
& \image{8}{img/visu/g-SiO2at}
& & & 
\image{8}{img/visu/g-SiO2cl}
\end{tabular}
\caption[The concept of {\bf{clones}} in the \atomes\ program.]{The concept of {\bf{clones}} in the \atomes\ program. {\bf{a)}} 
Standard representation using cylinders, {\bf{b)}} Clones are included in the representation.}\label{atcl}}
\end{figure}}

\newcommand{\mchemfig}{
\myfigure{h}{mchem}{
\hspace{-1.5cm}
\begin{tabular}{lccr}
 & {\bf{a)}} \\
 & & \multicolumn{2}{r}{\image{10}{img/visu/menu/chem/tc-show}} \\
 & & \multicolumn{2}{r}{\image{10}{img/visu/menu/chem/pc-show}} \\
 & {\bf{b)}} \\
\multirow{8}{5cm}{\image{5}{img/visu/menu/m_chem}} & & \multicolumn{2}{r}{\image{10}{img/visu/menu/chem/ptc-show}} \\
 & & \multicolumn{2}{r}{\image{10}{img/visu/menu/chem/ppc-show}} \\
 & & \multicolumn{2}{r}{\image{10}{img/visu/menu/chem/prings-show}} \\
 & {\bf{c)}} \\
 & & \multicolumn{2}{r}{\image{10}{img/visu/menu/chem/rings}} \\
 & {\bf{d)}} \\
 & & \multicolumn{2}{r}{\image{10}{img/visu/menu/chem/frag-show}} \\
 & {\bf{e)}} \\
 & &  \multicolumn{2}{r}{\image{10}{img/visu/menu/chem/mol-show}} 
\end{tabular}}
{The \aob{Chemistry} menu and the attached submenus.}
{The \aob{Chemistry} menu and the attached submenus.}}

\newcommand{\mtoolfig}{
\myfigure{h}{mtools}{
\hspace{-1.5cm}
\begin{tabular}{lccr}
 & {\bf{a)}} \\
 & & \multicolumn{2}{r}{\image{9.5}{img/visu/menu/tools/edit}} \\
 & {\bf{b)}} \\
 & & \multicolumn{2}{r}{\image{9.5}{img/visu/menu/tools/mode}} \\
\multirow{8}{5cm}{\image{5}{img/visu/menu/m_tools}} & {\bf{c)}} \\
 & & \multicolumn{2}{r}{\image{9.5}{img/visu/menu/tools/selection}} \\
 & {\bf{d)}} \\
 & & \multicolumn{2}{r}{\image{9.5}{img/visu/menu/tools/md}} \\
 & {\bf{e)}} \\
 & & \multicolumn{2}{r}{\image{9.5}{img/visu/menu/tools/invert}} \\
\end{tabular}}
{The \aob{Tools} menu and the attached submenus in the  \atomes\ program.}
{The \aob{Tools} menu and the attached submenus in the  \atomes\ program.}}

\newcommand{\mviewfig}{
\myfigure{h}{mview}{
\hspace{-1cm}
\begin{tabular}{lccr}
 & {\bf{a)}} \\
 & & \multicolumn{2}{r}{\image{9.5}{img/visu/menu/view/rep}} \\
 & {\bf{b)}} \\
 & & \multicolumn{2}{r}{\image{9.5}{img/visu/menu/view/proj}} \\
\multirow{8}{5cm}{\image{5}{img/visu/menu/m_view}} & {\bf{c)}} \\
 & & \multicolumn{2}{r}{\image{9.5}{img/visu/menu/view/back}} \\
 & {\bf{d)}} \\
 & & \multicolumn{2}{r}{\image{9.5}{img/visu/menu/view/axis}} \\
\end{tabular}}
{The \aob{View} menu and the attached submenus in the  \atomes\ program.}
{The \aob{View} menu and the attached submenus in the  \atomes\ program.}}

\newcommand{\oselfig}{
\myfigure{h}{vsel}{
%\hspace{-1cm}
\begin{tabular}{lclclc} 
{\bf{a)}} & &  & & {\bf{b)}} \\
 & \image{4}{img/visu/sel/l-GeS2-a} & & \raisebox{2cm}{$\Longrightarrow$} & &
   \image{4}{img/visu/sel/l-GeS2-b} \\
& & {\bf{c)}} \\
& \raisebox{2cm}{$\Nwarrow$} & &
\image{4}{img/visu/sel/l-GeS2-c} & & \raisebox{2cm}{$\Swarrow$} \\
\end{tabular}}
{Illustration of the selection / label process using the mouse left button.}
{Illustration of the selection / label process using the mouse left button.}}

\newcommand{\recfig}{
\myfigure{h}{rec}{
\begin{tabular}{lcp{1cm}lc} 
{\bf{a)}} & & & {\bf{b)}} \\
 & \image{4}{img/visu/record} & &
 & \image{4}{img/visu/recording}
\end{tabular}}
{The \aob{Recorder} dialog of the OpenGL window in the \atomes\ program.}
{The \aob{Recorder} dialog of the OpenGL window in the \atomes\ program.}}

\newcommand{\ocmfig}{
\myfigure{h}{mocm}{
\begin{tabular}{lcp{1cm}lc}
 {\bf{a)}} & & & {\bf{b)}} \\
 & \image{6}{img/visu/ocm/ocm-at} & &
 & \image{6}{img/visu/ocm/ocm-bd} \\
\end{tabular}}
{The object contextual menu: {\bf{a)}} for an atom, {\bf{b)}} for a chemical bond.}
{The object contextual menu: {\bf{a)}} for an atom, {\bf{b)}} for a chemical bond.}}

\newcommand{\csosfig}{
\myfigure{h}{mocmb}{
\begin{tabular}{lcp{1cm}c}
 \\
 {\bf{a)}} \\
 & \image{6}{img/visu/ocm/ocm-thisat} & 
 & \image{6}{img/visu/ocm/ocm-thisbd} \\
 {\bf{b)}} \\
 & \multicolumn{3}{c}{\image{6}{img/visu/ocm/ocm-sel}} \\
 & & or \\ 
 & \multicolumn{3}{c}{\image{6}{img/visu/ocm/ocm-nsel}} \\
 {\bf{c)}} \\
 & \multicolumn{3}{c}{\image{6}{img/visu/ocm/ocm-lab}} \\
 & & or \\
 & \multicolumn{3}{c}{\image{6}{img/visu/ocm/ocm-nlab}} \\
 {\bf{d)}} \\
 & \image{6}{img/visu/ocm/ocm-coord} &
 & \image{6}{img/visu/ocm/ocm-bcoord}		
\end{tabular}}
{Construction of the submenus attached to the object contextual menu.}
{Construction of the submenus attached to the object contextual menu.}}

% Chap 6

\newcommand{\emfig}{
\myfigure{h}{emf}{
\begin{tabular}{lclc}
{\bf{a)}} & & {\bf{b)}} \\
 & \image{8}{img/edit/edit}
 & & \image{8}{img/edit/mode}
\end{tabular}}
{Accessing the edition tools in the \atomes\ program.}
{Accessing the edition tools in the \atomes\ program.}}

\newcommand{\eselfig}{\myfigure{h}{v2sel}{
\begin{tabular}{lclclc} 
{\bf{a)}} & &  & & {\bf{b)}} \\
 & \image{4}{img/edit/atoms/sel-b1} & & \raisebox{2cm}{$\Longrightarrow$} & &
 \image{4}{img/edit/atoms/sel-b2} \\
& & {\bf{c)}} \\
& \raisebox{2cm}{$\Nwarrow$} & &
 \image{4}{img/edit/atoms/sel-b3} & & \raisebox{2cm}{$\Swarrow$} \\
\end{tabular}}
{Modified selection process when the \aob{Model edition} window is opened.}
{Modified selection process when the \aob{Model edition} window is opened.}}

\newcommand{\repnmfig}{
\myfigure{h}{repnm}{
\begin{tabular}{lclc}
\hspace{-1.0cm}{\bf{a)}} & & {\bf{b)}} \\
 & \hspace{-1.0cm} \image{8}{img/edit/atoms/replace/normally} &
 & \image{8}{img/edit/atoms/replace/random}
\end{tabular}
\begin{tabular}{lcl}
 & {\bf{c)}} \\
\multirow{8}{5cm}{\image{4.54}{img/edit/atoms/replace/replace_menu}} & 
 & \image{10}{img/edit/atoms/replace/replace_atoms} \\
 &  {\bf{d)}} \\
& & \image{10}{img/edit/atoms/replace/replace_lib} \\
 & {\bf{e)}} \\
& & \image{10}{img/edit/atoms/replace/replace_proj} \\
 & {\bf{f)}} \\
& & \image{10}{img/edit/atoms/replace/replace_copied}
\end{tabular}}
{Search trees and \aob{Select for ...} menu in the \aob{Replace} tab.}
{Top: search trees {\bf{a)}} replacing object(s) one by one, {\bf{b)}} replacing object(s) randomly. 
Bottom: the \aob{Select for ...} replacement menu and the attached {\bf{c)}} \aob{Atom}, {\bf{d)}} \aob{Library}, {\bf{e)}} \aob{Import from project} and {\bf{f)}} \aob{Copied data} submenus.}}

\newcommand{\editoolfig}{
\myfigure{h}{editool}{
\begin{tabular}{cc}
{\bf{a)}} \\
 & \image{10}{img/edit/mouse/edition}
\end{tabular}
\begin{tabular}{cccc}
{\bf{b)}} \\
 & \image{7}{img/edit/mouse/am} & \raisebox{0.75cm}{$\Longrightarrow$} &
\image{7}{img/edit/mouse/em}
\end{tabular}}
{Activating the mouse \aob{Edition} mode using the \aob{Tools} menu.}
{{\bf{a)}} Activating the mouse \aob{Edition} mode using the \aob{Tools} menu. {\bf{b)}} when switching mode the title bar of the OpenGL window changes.}}

\newcommand{\rccmfig}{
\myfigure{h}{rccm}{
\begin{tabular}{lccr}
\multirow{4}{6.5cm}{\vspace{0.5cm} \image{6.5}{img/edit/mouse/rc_cm}} & {\bf{a)}} \\
 & &
 \multicolumn{2}{r}{\image{9.5}{img/edit/mouse/rc_tools}} \\
 &  {\bf{b)}} \\
& & \multicolumn{2}{r}{\image{9.5}{img/edit/mouse/rc_insert}}
\end{tabular}}
{The right button contextual menu on the background of the OpenGL window.}
{The right button contextual menu on the background of the OpenGL window.}}

\newcommand{\oecmfig}{
\myfigure{h}{oecm}{
\begin{tabular}{lcp{1cm}lc}
 {\bf{a)}} & & & {\bf{b)}} \\
 & \image{6}{img/edit/mouse/ocm-atom} & &
 & \image{6}{img/edit/mouse/ocm-bond} \\
\end{tabular}}
{The object edition contextual menu: {\bf{a)}} for an atom, {\bf{b)}} for a chemical bond.}
{The object edition contextual menu: {\bf{a)}} for an atom, {\bf{b)}} for a chemical bond.}}

% Appendix 4

\newcommand{\spheresfig}{
\myfigure{h}{spheres}{
\begin{tabular}{cccc}
\image{2}{img/bonds/SGe} \hspace{1cm} &
\image{2}{img/bonds/GeS3} \hspace{1cm} & 
\image{2}{img/bonds/GeS4} \hspace{1cm} & 
\image{2}{img/bonds/GeS3Ge} \\
\image{2}{img/bonds/SGe2} \hspace{1cm} &
\image{2}{img/bonds/SGe2S} \hspace{1cm} &
\image{2}{img/bonds/SSGe}
\end{tabular}}
{Illustration of several coordination spheres that can be found in glassy GeS$_2$.}
{Illustration of several coordination spheres that can be found in glassy GeS$_2$.}}

\newcommand{\cornedgefig}{
\myfigure{h}{cornedges}{
\image{4}{img/bonds/corner}
\hspace{2cm}
\image{4}{img/bonds/edge}}
{Corner sharing (left) and edge sharing (right) tetrahedra.}
{Corner sharing (left) and edge sharing (right) tetrahedra.}}

\newcommand{\danglesfig}{
\myfigure{h}{dangles}{
\image{4}{img/bonds/angles}
\hspace{2cm}
\image{4}{img/bonds/dihedral}}
{Bond angles (left) and diehdral angles (right).}
{Bond angles (left) and diehdral angles (right).}}

\newcommand{\rnrcsitab}{
\mytable{h}{rnrcsi}{
\begin{minipage}{5cm}
\begin{tabular}{c|p{2cm}}
$n$ & \rpn \\
\hline
3 & 1/10 \\
4 & 1/10 
\end{tabular}
\end{minipage}
\hspace{1cm}
\begin{minipage}{5cm}
\begin{tabular}{c|p{2cm}}
$n$ & \rpc \\
\hline
3 & 3/10 \\
4 & 4/10
\end{tabular}
\end{minipage}}
{$R_N$ and $R_C$ for the network in figure~\ref{exering}}{$R_N$ and $R_C$ for the network in figure~\ref{exering}}}

\newcommand{\nabnantab}{
\mytable{h}{nabnan}{
\begin{minipage}{5cm}
\begin{tabular}{c|p{2cm}}
$n$ & \rpn \\
\hline
3 & 1/10 \\
4 & 1/10 
\end{tabular}
\end{minipage}
\hspace{1cm}
\begin{minipage}{5cm}
\begin{tabular}{c|p{2cm}}
$n$ & \rpc \\
\hline
3 & 3/10 \\
4 & 4/10
\end{tabular}
\end{minipage}}
{$R_N$ and $R_C$ calculated for the networks illustrated in figure~\ref{s34}.}
{$R_N$ and $R_C$ calculated for the networks illustrated in figure~\ref{s34}.}}

\newcommand{\nabtqdtab}{\mytable{h}{34-2NAB}{
\begin{minipage}{5cm}
\begin{tabular}{c|p{2cm}}
$n$ & \rpc \\
\hline
3 & 1/10 \\
4 & 1/10 
\end{tabular}
\end{minipage}
\hspace{1cm}
\begin{minipage}{5cm}
\begin{tabular}{c|p{2cm}}
$n$ & \rpc \\
\hline
3 & 0/10 \\
4 & 2/10
\end{tabular}
\end{minipage}}
{Number of rings in the simple networks represented in figure~\ref{s34-2}.}
{Number of rings in the simple networks represented in figure~\ref{s34-2}.}}

\newcommand{\gmatrix}{\mytable{h}{matmn}{
\begin{tabular}{c}
C$_{mat}$ =
$ \left[ \begin{array}{cccc} 
P_N(r) & P_N(r+1,r) & \cdots & P_N(R,r) \\
P_N(r,r+1) & \ddots &        & P_N(R,r+1) \\
\vdots   &          & \ddots & \vdots  \\
P_N(r,R) &   \cdots & \cdots & P_N(R) 
\end{array} \right] $
\end{tabular}}
{General connectivity matrix.}{General connectivity matrix.}}


\newcommand{\kbdmain}{
\begin{minipage}{8cm}
\begin{itemize}
\item Workspace: 
\item[] \Ctrl + \keystroke{w} : open workspace
\item[] \Ctrl + \keystroke{s} : save workspace as
\item[] \Ctrl + \keystroke{c} : close workspace \\
\item Project:
\item[] \Ctrl + \keystroke{n} : create new project
\item[] \Ctrl + \keystroke{o} : open project
\end{itemize}
\end{minipage}
\begin{minipage}{8cm}
\begin{itemize}
\item Misc:
\item[] \Ctrl + \keystroke{t} : show curve toolboxes
\item[] \Ctrl + \keystroke{p} : open periodic table
\item[] \Ctrl + \keystroke{a} : show about dialog
\item[] \Ctrl + \keystroke{q} : quit \\
\item On any file dialog:
\item[] \Ctrl + \keystroke{l} : open command line
\end{itemize}
\end{minipage}}

\newcommand{\kbdcurve}{
\begin{itemize}
\item Combined keys shortcuts:
\item[] \Ctrl + \keystroke{a} : Autoscale
\item[] \Ctrl + \keystroke{c} : Close curve window 
\item[] \Ctrl + \keystroke{e} : Open the data plot editing tool box [Fig.~\ref{edittool}]
\item[] \Ctrl + \keystroke{i} : Export image
\item[] \Ctrl + \keystroke{s} : Save / export data
\end{itemize}
}

\newcommand{\kbdviz}{
\begin{itemize}
\item Single key shortcuts:
\begin{itemize}
\item Colors:
\item[] \keystroke{a} : change atom(s) colormap
\item[] \keystroke{p} : change polyhedra(ons) colormap \\
\item Styles: 
\item[] \keystroke{b} : change default style to \aob{Ball and stick}  
\item[] \keystroke{c} : change default style to \aob{Cylinders}
\item[] \keystroke{d} : change default style to \aob{Dots} 
\item[] \keystroke{s} : change default style to \aob{Spheres} 
\item[] \keystroke{o} : change default style to \aob{Covalent radius}
\item[] \keystroke{i} : change default style to \aob{Ionic radius}
\item[] \keystroke{v} : change default style to \aob{van Der Waals radius}
\item[] \keystroke{r} : change default style to \aob{In cristal radius}
\item[] \keystroke{w} : change default style to \aob{Wireframe} \\
\item Measures: 
\item[] \keystroke{m} : show all measures for the selection, if pressed:
\begin{itemize}
\item once: display inter-atomic distance(s)
\item twice: display inter-atomic angles
\item a third time: hide measures \\
\end{itemize}
\item Model rotation:
\item[] \RArrow : rotate right
\item[] \LArrow : rotate left
\item[] \UArrow : rotate up
\item[] \DArrow : rotate down \\
\end{itemize}
\item Misc:
\item[] \Esc : exit fullscreen mode
\item[] \Spacebar : pause / restart spinning
%\item[] \Del : Delete all selected atom(s)
\newpage
\item Combined keys shortcuts:
\begin{itemize}
\item Mouse mode:
\item[] \Alt + \keystroke{a} : enter mouse \aob{Analysis} mode
\item[] \Alt + \keystroke{e} : enter mouse \aob{Edition} mode \\
\item Selection:
\item[] \Ctrl + \keystroke{a} : select / unselect all atoms
\item[] \Ctrl + \keystroke{c} : copy all selected atom(s)
%\item[] \Ctrl + \keystroke{v} : paste copied selection (if the model is not a MD trajectory)
%\item[] \Ctrl + \keystroke{x} : copy, then delete selection
%\item[] \Ctrl + \keystroke{n} : create new (empty project) \\
\item Mouse selection mode:
\item[] \Shift + \keystroke{a}: atom selection mode
\item[] \Shift + \keystroke{c}: coordination sphere selection mode
\item[] \Shift + \keystroke{f}: fragment selection mode
\item[] \Shift + \keystroke{m}: molecule selection mode \\	
\item Camera motion:
\item[] \Ctrl + \RArrow : move camera right
\item[] \Ctrl + \LArrow : move camera left
\item[] \Ctrl + \UArrow : move camera up
\item[] \Ctrl + \DArrow : move camera down
\item[] \Shift + \UArrow : zoom out
\item[] \Shift + \DArrow : zoom in \\
\item Spinning: 
\item[] \Ctrl + \Shift + \RArrow : spin right / increase speed right or reduce speed left
\item[] \Ctrl + \Shift + \RArrow : spin left / increase speed left or reduce speed right
\item[] \Ctrl + \Shift + \UArrow : spin up / increase speed up or reduce speed down
\item[] \Ctrl + \Shift + \DArrow : spin down / increase speed down or reduce speed up
\item[] \Ctrl + \keystroke{s} : stop spinning \\
\item Misc:
\item[] \Ctrl + \keystroke{l} : label / unlabel all atoms 
\item[] \Ctrl + \keystroke{e} : open the \aob{Environments configuration} window [Sec.~\ref{ecw}]
\item[] \Ctrl + \keystroke{m} : open the \aob{Measures} dialog [Sec.~\ref{mdw}]
\item[] \Ctrl + \keystroke{r} : open the \aob{Recorder} dialog [Sec.~\ref{rdw}]
\item[] \Ctrl + \keystroke{f} : enter / exit fullscreen mode 
\end{itemize}
\end{itemize}}

\newcommand{\kbdedit}{
\begin{itemize}
\item Single key shortcuts:
\begin{itemize}
\item Colors:
\item[] \keystroke{a} : change atom(s) colormap
\item[] \keystroke{p} : change polyhedra(ons) colormap \\
\item Styles: 
\item[] \keystroke{b} : change default style to \aob{Ball and stick}
\item[] \keystroke{c} : change default style to \aob{Cylinders}
\item[] \keystroke{d} : change default style to \aob{Dots}
\item[] \keystroke{s} : change default style to \aob{Spheres}
\item[] \keystroke{o} : change default style to \aob{Covalent radius}
\item[] \keystroke{i} : change default style to \aob{Ionic radius}
\item[] \keystroke{v} : change default style to \aob{van Der Waals radius}
\item[] \keystroke{r} : change default style to \aob{In cristal radius}
\item[] \keystroke{w} : change default style to \aob{Wireframe}
\item Measures: 
\item[] \keystroke{m} : show all measures for the selection, if pressed:
\begin{itemize}
\item once: display inter-atomic distance(s)
\item twice: display inter-atomic angles
\item a third time: hide measures \\
\end{itemize}
\item Atomic coordinates rotation:
\item[] \RArrow : rotate atomic coordinates right
\item[] \LArrow : rotate atomic coordinates left
\item[] \UArrow : rotate atomic coordinates up
\item[] \DArrow : rotate atomic coordinates down \\
\item Misc:
\item[] \Esc : exit fullscreen mode
\item[] \Del : Delete all selected atom(s)
\end{itemize}
\newpage
\item Combined keys shortcuts:
\begin{itemize}
\item Mouse mode:
\item[] \Alt + \keystroke{a} : enter mouse \aob{Analysis} mode
\item[] \Alt + \keystroke{e} : enter mouse \aob{Edition} mode \\
\item Selection:
\item[] \Ctrl + \keystroke{a} : select / unselect all atoms
\item[] \Ctrl + \keystroke{c} : copy all selected atom(s)
\item[] \Ctrl + \keystroke{v} : paste copied selection (if the model is not a MD trajectory)
\item[] \Ctrl + \keystroke{x} : copy, then delete selection
\item[] \Ctrl + \keystroke{n} : create new (empty project) \\
\item Atomic coordinates translation:
\item[] \Ctrl + \RArrow : translate atomic coordinates right
\item[] \Ctrl + \LArrow : translate atomic coordinates left
\item[] \Ctrl + \UArrow : translate atomic coordinates up
\item[] \Ctrl + \DArrow : translate atomic coordinates down
\item[] \Shift + \UArrow : zoom out
\item[] \Shift + \DArrow : zoom in \\
\item Misc:
\item[] \Ctrl + \keystroke{l} : label / unlabel all atoms 
\item[] \Ctrl + \keystroke{e} : open the \aob{Environments configuration} window [Sec.~\ref{ecw}]
\item[] \Ctrl + \keystroke{m} : open the \aob{Measures} dialog [Sec.~\ref{mdw}]
\item[] \Ctrl + \keystroke{r} : open the \aob{Recorder} dialog [Sec.~\ref{rdw}]
\item[] \Ctrl + \keystroke{f} : enter / exit fullscreen mode 
\end{itemize}
\end{itemize}}

\newcommand{\allkbd}{
\begin{itemize}
\item Main window
\begin{itemize}
\item Workspace: \\
\item[] \Ctrl + \keystroke{w} : open workspace
\item[] \Ctrl + \keystroke{s} : save workspace as
\item[] \Ctrl + \keystroke{c} : close workspace \\
\item Project: \\
\item[] \Ctrl + \keystroke{n} : create new project
\item[] \Ctrl + \keystroke{o} : open project \\
\item Misc: \\
\item[] \Ctrl + \keystroke{t} : show curve toolboxes
\item[] \Ctrl + \keystroke{p} : open periodic table
\item[] \Ctrl + \keystroke{a} : show about dialog
\item[] \Ctrl + \keystroke{q} : quit
\end{itemize}
\newpage
\item Curve window: \\
\item[] \Ctrl + \keystroke{a} : Autoscale
\item[] \Ctrl + \keystroke{c} : Close curve window 
\item[] \Ctrl + \keystroke{e} : Open the data plot editing tool box [Fig.~\ref{edittool}]
\item[] \Ctrl + \keystroke{i} : Export image
\item[] \Ctrl + \keystroke{s} : Save / export data \\
\item OpenGL window:
\begin{itemize}
\item Single key shortcuts: \\
\begin{itemize}
\item Colors: \\
\item[] \keystroke{a} : change atom(s) colormap
\item[] \keystroke{p} : change polyhedra(ons) colormap \\
\item Styles:  \\
\item[] \keystroke{b} : change default style to \aob{Ball and stick}
\item[] \keystroke{c} : change default style to \aob{Cylinders}
\item[] \keystroke{d} : change default style to \aob{Dots}
\item[] \keystroke{s} : change default style to \aob{Spheres}
\item[] \keystroke{o} : change default style to \aob{Covalent radius}
\item[] \keystroke{i} : change default style to \aob{Ionic radius}
\item[] \keystroke{v} : change default style to \aob{van Der Waals radius}
\item[] \keystroke{r} : change default style to \aob{In cristal radius}
\item[] \keystroke{w} : change default style to \aob{Wireframe} \\
\item Measures:  \\
\item[] \keystroke{m} : show all measures for the selection, if pressed: \\
\begin{itemize}
\item once: display inter-atomic distance(s)
\item twice: display inter-atomic angles
\item a third time: hide measures \\
\end{itemize}
\item Misc: \\
\item[] \Esc : exit fullscreen mode
\item[] \Spacebar : pause / restart spinning \\
\end{itemize}
\item Combined keys shortcuts: \\
\begin{itemize}
\item Mouse mode: \\
\item[] \Alt + \keystroke{a} : enter mouse \aob{Analysis} mode
\item[] \Alt + \keystroke{e} : enter mouse \aob{Edition} mode \\
\item Selection: \\
\item[] \Ctrl + \keystroke{a} : select / unselect all atoms
\item[] \Ctrl + \keystroke{c} : copy all selected atom(s)
%\item[] \Ctrl + \keystroke{v} : paste copied selection (if the model is not a MD trajectory)
%\item[] \Ctrl + \keystroke{x} : copy, then delete selection
\item[] \Ctrl + \keystroke{n} : create new (empty project) \\
\item Misc: \\
\item[] \Ctrl + \keystroke{l} : label / unlabel all atoms 
\item[] \Ctrl + \keystroke{e} : \aob{Environments configuration} window [Sec.~\ref{ecw}]
\item[] \Ctrl + \keystroke{m} : \aob{Measures} dialog [Sec.~\ref{mdw}]
\item[] \Ctrl + \keystroke{r} : \aob{Recorder} dialog [Sec.~\ref{rdw}]
\item[] \Ctrl + \keystroke{f} : enter / exit fullscreen mode \\ 
\item Camero motion: \\
\item[] \Shift + \UArrow : zoom out
\item[] \Shift + \DArrow : zoom in \\
\end{itemize}
\end{itemize}
\item OpenGL window {\bf{Analysis mode only}}
\begin{itemize}
\item Single key shortcuts: \\
\begin{itemize}
\item Model rotation: \\
\item[] \RArrow : rotate right
\item[] \LArrow : rotate left
\item[] \UArrow : rotate up
\item[] \DArrow : rotate down \\
\end{itemize}
\newpage
\item Combined keys shortcuts: \\
\begin{itemize}
\item Camera motion: \\
\item[] \Ctrl + \RArrow : move camera right
\item[] \Ctrl + \LArrow : move camera left
\item[] \Ctrl + \UArrow : move camera up
\item[] \Ctrl + \DArrow : move camera down \\
\item Spinning: \\
\item[] \Ctrl + \Shift + \RArrow : spin right / increase speed r. or reduce speed left
\item[] \Ctrl + \Shift + \RArrow : spin left / increase speed left or reduce speed right
\item[] \Ctrl + \Shift + \UArrow : spin up / increase speed up or reduce speed down
\item[] \Ctrl + \Shift + \DArrow : spin down / increase speed d. or reduce speed up \\
\item[] \Ctrl + \keystroke{s} : stop spinning \\
\end{itemize}
\end{itemize}
\item OpenGL window: {\bf{Edition mode only}}
\begin{itemize}
\item Single key shortcuts: \\
\begin{itemize}
\item Atomic coordinates rotation: \\
\item[] \RArrow : rotate atomic coordinates right
\item[] \LArrow : rotate atomic coordinates left
\item[] \UArrow : rotate atomic coordinates up
\item[] \DArrow : rotate atomic coordinates down \\
\end{itemize}
\item Combined keys shortcuts: \\
\begin{itemize}
\item Atomic coordinates translation: \\
\item[] \Ctrl + \RArrow : translate atomic coordinates right
\item[] \Ctrl + \LArrow : translate atomic coordinates left
\item[] \Ctrl + \UArrow : translate atomic coordinates up
\item[] \Ctrl + \DArrow : translate atomic coordinates down
\end{itemize}
\end{itemize}
\end{itemize}}

